% !TeX spellcheck = de_CH
%!TEX root = ./thesis.tex

\addcontentsline{toc}{section}{Kurzfassung} 
\chapter*{Kurzfassung}

Die Entwicklung von räumlichen Eigenmoden und die Analyse ihres zeitlichen Verhaltens ist das grundlegende Ziel vieler Bereichen der modernen Physik. Insbesondere stark angetriebene, dissipative Moden spielen in heutiger experimentellen Forschung eine zunehmende Rolle, auch wenn ihre nichtlinearen Merkmale eine erhebliche Herausforderung darstellen. Die in dieser Arbeit verfolgten Ziele sind zweierlei. (i) Die Fähigkeit zur gezielten Schaffung von Moden mit erwünschten Eigenschaften durch die Entwicklung eines geeigneten Mediums. (ii) Genaue Analyse ihrer zeitlichen Dynamik über das lineare Verhalten hinaus. 

Im ersten Teil dieser Arbeit untersuchen wir die Entstehung von Eigenmoden in klassischen Metamaterialien mithilfe der Topologie. Das zugrunde liegende Konzept ist Symmetrie. Zunächst setzen wir die Symmetrien eines gemusterten Materials mit seiner Bandstruktur in Beziehung und analysieren die Folgen ihrer Brechung. Anschliessend leiten wir aus der Gruppentheorie allgemeine Argumente her, die eine vollständige Liste der möglichen Symmetriebrechungsmechanismen ergeben. Diese können verwendet werden, um topologische Randmoden zu konstruieren, die am Rand von Materialien mit entsprechend gebrochenen Symmetrien aufkommen. Wir zeigen, dass durch konsequente Verwendung der Gruppentheorie eine Topologie zweiter Ordnung aufgebaut werden kann, die lokalisierte Moden in zweidimensionalen Medien ergibt. Solche Moden weisen hohe Qualitätsfaktoren und topologisch bestimmte Entartung auf, was sie zu einem vielversprechenden Kandidaten für Halbleiterlaser macht.

Im zweiten Teil untersuchen wir das zeitliche Verhalten der Moden unter zeitabhängiger Anregung. Ist diese genügend stark, verhalten sich die Moden nichtlinear und werden durch nichtlineare Differentialgleichungen beschrieben. Wir verwenden die \textit{harmonic balance} Methode, um unser Problem in den Fourierraum zu transformieren, wo die stationären Lösungen des Systems als Wurzeln multivariater Polynome erscheinen. Diese erhalten wir mithilfe der \textit{homotopy continuation} Methode, die es garantiert, alle Wurzeln zu finden. Auf diese Weise können wir bisher schwer lösbare Systeme wie z. B. zwei gekoppelte angeregte nichtlineare Oszillatoren oder einen einzelnen multiharmonischen Oszillator angehen. Anschliessend formulieren wir Kriterien für die Stabilität der Lösungen und deren Reaktion auf kleine Störungen wie Rausch oder ein zusätzliches Signal. Die charakteristischen nichtlinearen Phänomene, wie Multistabilität und Frequenzumwandlung, werden so in einen kohärenten Rahmen gesetzt.

Mit den Lösungen und ihrer Stabilität in der Hand entwickeln wir eine Methodik für die Behandlung von Grenzzyklen – aperiodischen Lösungen von periodisch angeregten Systemen. Diese lassen sich durch \textit{harmonic balance} ebenso gut beschreiben, weisen aber eine unendliche Entartung der stationären Lösungen auf, was numerische Arbeit erheblich erschwert. Durch die Feststellung und Beseitigung dieser Entartung ermöglichen wir die Untersuchung von Grenzzyklen durch \textit{harmonic balance}, die wir am Beispiel von angetriebenen nichtlinearen gekoppelten Oszillatoren demonstrieren. Eine weitere Erweiterung besteht darin, unseren Rahmen mit dem Formalismus der Quantenmechanik in Einklang zu bringen, wo \textit{harmonic balance} als die sogennante Drehwellenannäherung bekannt ist. Wir analysieren deren Unfähigkeit, die Lösung eines klassischen Oszillators genau zu bestimmen, und leiten eine Methode zur Beseitigung dieses Problems her. Schliesslich wird ein konkretes System genauer analysiert: parametrisch gekoppelte Oszillatoren und ihre Verwendung als Sensor in der Magnetresonanztomographie im Nanobereich. Wir entwickeln ein Messprotokoll und bestimmen dessen erwartete Leistung.

Das übergreifende Ergebnis dieser Bemühungen ist die Erstellung eines Open-Source Software-Paket, HarmonicBalance.jl. Das Paket enthält sowohl die nachgezählten als auch die neu entwickelten Methoden, und soll damit als Werkzeug für künftige Forschung nichtlinearer harmonischer Systeme dienen. 
