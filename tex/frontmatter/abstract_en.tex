% !TeX spellcheck = en_GB
%!TEX root = ./thesis.tex

\addcontentsline{toc}{section}{Abstract} 
\chapter*{Abstract}

Engineering spatial modes and analysing their temporal behaviour is the underlying challenge of many fields of physics. In particular, strongly-driven dissipative modes are increasingly present in contemporary experimental research, despite the challenge posed by their intricate nonlinear properties. The goals pursued in this thesis are twofold. (i) The ability to design modes with desirable characteristics by engineering a suitable medium. (ii) Accurately analysing their behaviour in time, without restriction to the linear limit.

In the first part of this thesis, we investigate the creation of modes in classical metamaterials using topology. The underlying theme is \textit{symmetry}. We first relate the symmetries of a patterned material to its band structure and analyse the effects of their breaking. We then formulate general arguments based on group theory to obtain an exhaustive list of the possible symmetry-breaking mechanisms. These can be used to construct topological boundary modes, which are guaranteed to arise at the interface of bulk materials with suitably-broken symmetries. We show that using the group-theoretical framework, \textit{second-order topology} can be induced, creating point-like modes in a two-dimensional medium. Such modes possess high quality factors and topologically determined multiplicity, making them a promising candidate for semiconductor lasers.

In the second part, we investigate the temporal behaviour of modes when subject to time-dependent driving. When driven strongly, these behave nonlinearly and are described by nonlinear differential equations. We use the method of harmonic balance to translate our problem into Fourier space, viewing each mode as a finite sum of Fourier components. Within this approach, steady-state responses of the system appear as roots of multivariate polynomials. We find these using the method of homotopy continuation, which is guaranteed to find all the roots. This allows us to solve hitherto intractable systems, such as two coupled driven nonlinear oscillators and a single multi-harmonic oscillator. We then formulate criteria for solution stability and response to small perturbations such as noise or an applied probe. Signature nonlinear phenomena, including multistability and frequency conversion, are thus brought into a coherent framework.

With the solutions and their stability at hand, we extend our methodology to treat limit cycles – instances of aperiodic response in periodically-driven systems. These yield well to harmonic balance but show an infinite degeneracy of steady states, prohibiting their efficient solving. Identifying and removing this degeneracy, we enable the study of limit cycles by the homotopy continuation method. The efficiency of this approach is demonstrated by characterising the limit cycle response in driven nonlinear coupled oscillators. Another extension is reconciling our framework with the formalism of quantum mechanics, where harmonic balance appears in the guise of the \textit{rotating-wave approximation}. We analyse its failure to faithfully reproduce the response of a classical oscillator and derive a correction to remedy the issue. Finally, one harmonic system is analysed in more detail: parametrically-coupled oscillators and their use as a sensor in nanoscale magnetic resonance imaging. We develop a measurement protocol and calculate its expected performance.

The overarching outcome of these efforts is the creation of an open-source software package, HarmonicBalance.jl. Incorporating both the recounted and newly-developed methods, the package is envisioned to serve as a powerful tool for future research of nonlinear harmonic systems. 

 

 