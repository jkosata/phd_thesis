%!TEX root = ../thesis.tex

\chapter*{Conclusions and outlook}
\addcontentsline{toc}{chapter}{Conclusions and outlook}

We have studied the spatial origins and temporal behaviour of modes in nonlinear media. While many example systems were presented throughout, the principal focus (Chapters \ref{ch:symmetry}-\ref{ch:rwa}) was on developing a consistent framework for treating the problems we encountered. A series of detailed system studies is the natural next step for the work presented here, with the potential of yielding applicable mechanisms for sensing and other purposes or altogether new, unexpected phenomena. 

Part \ref{part:spatial} dealt with engineering topological modes in classical metamaterials. In Chapter \ref{ch:symmetry}, we developed an approachable framework to derive the existence and properties of Dirac cones in continuous media. Beyond yielding a vast space of potential realisations of first-order topology, this enabled us to create second-order topological modes from purely symmetry-based arguments by mimicking the Jackiw-Rossi model in Chapter \ref{ch:hoti}. These have a range of potential applications, from semiconductor lasers to highly sensitive measurement devices. 

Turning to the temporal behaviour of driven modes, in Part \ref{part:temporal} we introduced the method of harmonic balance as the starting point for treating harmonically-driven nonlinear systems. Reducing the problem of finding steady states to solving coupled polynomial equations, we applied the method of homotopy continuation to obtain their roots. The combination of the two methods proved to be very powerful, successfully characterizing systems where hundreds of thousands of steady states were admissible. In Chapter \ref{ch:linresp}, we extended the study of steady states to consider their perturbation by external forces. This allowed us to formulate criteria for solution stability, as well as describe and numerically verify the effect of probe frequency conversion. In Chapter \ref{ch:hopf}, we relaxed the assumption of time-translation symmetry, which enabled us to adapt harmonic balance to compute the properties of limit cycles. Extending the formalism from classical to quantum, in Chapter \ref{ch:rwa} we discussed the relationship between the rotating-wave approximation and harmonic balance, highlighting the potential pitfalls when quantising driven-dissipative systems. Finally, in Chapter \ref{ch:spins}, we refocused from method development to a particular system, showing the suitability of parametric frequency conversion in spatially-patterned coupled modes as a sensing protocol for NanoMRI. 

Throughout Part \ref{part:temporal}, many references were made to HarmonicBalance.jl, which has considerable potential for further development. The core functionality and documentation are in place (Appendix \ref{app:hb}). However, advanced features, such as the various levels of approximation of linear response (Chapter \ref{ch:linresp}) and the gauge-fixed ansatz for limit cycles (Chapter \ref{ch:hopf}) go beyond the standard treatment found in contemporary work. Accessible documentation and examples should be made available to ensure broader adoption of these methods. Perhaps the most important is the missing support for quantum operator input. As we saw in Chapter \ref{ch:rwa}, calculating mean-field observables in the quantum formalism leads to a set of nonlinear ODEs fully compatible with harmonic balance and the associated framework developed in this thesis. However, the distinction between the quantum and classical languages creates an artificial gap between the respective research communities. Implementation of input methods for quantised variables is foreseen in the near future -- this will make the package accessible to researchers working in quantum optics, quantum electronics and other fields, greatly expanding its potential user base. 